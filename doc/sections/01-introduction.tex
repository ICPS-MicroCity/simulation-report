\section{Introduction}\label{sec:introduction}

Usually, in order to analyze and understand a system, the traditional scientific method suggests observing a phenomena and possibly measuring it under controlled conditions.
This workflow allows the observers to formulate hypothesis and then either validate them or refute them.
Nevertheless, it is not always possible to observe a system either for practical or ethical reasons or maybe because the system is too complex, or it does not exist.
This is the reason why \textbf{simulation} is introduced as a new way for describing scientific theories.
It is defined as the process with which we can study the dynamic evolution of a model system, usually through computational tools~\cite{parisi_2001}.

As already discussed in the~\href{https://github.com/ICPS-MicroCity/case-study-analysis/releases/tag/v0.2.1}{case-study-analysis}, the aim of this study is to determine whether a \textit{situated recommendation system} can positively affect the visitor flow in an amusement park or not.
Since the desired system does not exist, and it would be too costly to implement just for the sake of a proof of concept, the only way to validate the latter theory is through a simulation.

Building a simulation also allows working with a \textit{model} which is a representation and abstraction of the actual system.
This also means that the model will be simpler and will only contain those aspects that are strictly essential for the sake of the study that will be made.
