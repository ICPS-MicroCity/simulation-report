\section{The Case Study}\label{sec:the-case-study}

From now on, the terms of the \textit{ubiquitous language} identified in the analysis document will be used to describe the desired model;
please, refer to~\href{https://github.com/ICPS-MicroCity/case-study-analysis/releases/tag/v0.2.1}{case-study-analysis} to avoid ambiguities.

As already discussed in the above analysis, attractions may cause considerable queues with long waiting times.
The aim of this simulation is to determine whether a \textit{situated recommendation system} may help reduce the waiting time needed to benefit from an attraction.
In order to achieve this, two simulations will be developed:
\begin{enumerate}
    \item \textbf{Random Redirection}: this one should reflect the current system, that is, once a visitor is satisfied by an attraction they move towards another attraction which is statistically chosen randomly.
    \item \textbf{Recommended Redirection}: this one should reflect the desired system, that is, once a visitor is satisfied by an attraction they move towards the next one accepting a recommendation.
    The latter may suggest an attraction if it has a short queue and if it is close enough to the visitor.
\end{enumerate}
%
Specifically, the model of the two simulations will include the following aspects:
\begin{itemize}
    \item A real world map of the amusement park, that is \href{https://www.mirabilandia.it/}{\textit{Mirabilandia}}, one of the most famous amusement parks in Italy.
    In particular, the simulation will only focus on the environment inside the park's boundaries.
    Moreover, the map will contain a network of routes that connect the attractions and can be exploited by visitors to move around the park.
    \item Attractions of many types (rides, water slides, restaurants, etc.) physically situated on the map.
    Their location will adhere to the actual location inside the park.
    Finally, they will be able to dequeue a specific amount of visitors, that had previously enqueued, and satisfy them.
    \item Visitors that will be able to move towards a specific attraction, chosen with one of the above policies, and wait for their turn in order to benefit from the attraction itself.
    Once they have been satisfied, they will proactively choose the next attraction.
\end{itemize}
%Specifically, an amusement park populated by visitors moving around can be seen as a complex system where people direct themselves towards a specific location in order to satisfy their needs.
%Under this perspective, the \textit{Alchemist Simulator}~\cite{Pianini_2013} is perfect to endow simulated people with a personal will to satisfy their needs and direct themselves towards the desired attraction in the amusement park.
