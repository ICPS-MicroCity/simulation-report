\usepackage{hyperref}\section{The Simulation}

In this section, we discuss the main elements of the simulation. To fully understand the concepts that are discussed here, such as nodes and programs, please refer to the \href{https://alchemistsimulator.github.io/}{Alchemist documentation}.

\subsection{Map Environment}
The simulation concerns an existing geographical location, that is the amusement park of \href{https://www.mirabilandia.it/}{\textit{Mirabilandia}}. For this reason, the environment of the simulation must be a map featuring existing paths in the real world. To achieve this, \textit{Alchemist} allows to use maps provided by \href{https://www.openstreetmap.org/}{\textit{OpenStreetMap}}, the free wiki world map. \textit{OpenStreetMap} provides navigation capabilities on the whole planet; such data, weighs about 50GB, thus it is recommended to use an extract with the data relative to the interested area. One great way to obtain an extract is through \href{https://extract.bbbike.org/}{\textit{BBBike}} \cite{Pianini_2013}.

Once extracted the map in the \texttt{.pbf} format, it is possible to build the simulation environment through the \href{https://alchemistsimulator.github.io/reference/kdoc/alchemist/it.unibo.alchemist.model.implementations.environments/-o-s-m-environment/}{\texttt{OSMEnvornment}}, as shown in the following listing (\ref{code:osm}).

\begin{lstlisting}[language=yaml, label=code:osm, caption=Building an \textit{OpenStreetMap} environment.]
# Use an OpenStreetMap Environment,
# deploying nodes only on streets.
environment:
  type: OSMEnvironment
  parameters: [mirabilandia.osm.pbf, true]
\end{lstlisting}

\noindent
The constructor of the environment accepts two parameters:
\begin{itemize}
    \item \texttt{file: String}, the path to the file containing the exported map;
    \item \texttt{onlyOnStreets: Boolean}, a boolean value allowing to deploy nodes only on the streets.
\end{itemize}

\subsection{Deployed Nodes}
After setting the environment, it is essential to deploy \textbf{nodes} on the map. In the current simulation, the elements that are represented by nodes are:
\begin{itemize}
    \item \textbf{Visitors}, as single individuals or groups; the key point here is that a node should correspond to one or more people using a single wearable device that tracks and guides its carrier.
    \item \textbf{Attractions}, that can be of different types, such as rides, water slides, restaurants, etc.; they are considered \textit{rendesvouz} points for visitors and are made of several sensors that allow to keep track of many information, such as the number of people waiting in a queue.
\end{itemize}

\noindent
In order to deploy nodes on the map it is mandatory to declare them under the \texttt{deployment} keyword, as shown in the following listing.

\begin{lstlisting}[language=yaml, label=code:deployment, caption=Deploying 1 attraction and 100 visitors inside the \texttt{bounds} polygon.]
# Define visitors
_visitors: &visitors
  - type: Polygon
    parameters: [ 100, *bounds ]
    contents:
      - { molecule: visitor, concentration: true }

# Define attractions
_attractions: &attractions
  - type: Point
    parameters: [44.33589, 12.26293]
    contents:
      - { molecule: attraction, concentration: true }
      - { molecule: attractionType, concentration: "\"restaurant\"" }
      - { molecule: capacity, concentration: 10 }
      - { molecule: name, concentration: "\"McDonald's\"" }

# Deploy nodes
deployments:
  - *attractions
  - *visitors
\end{lstlisting}

\noindent
Moreover, it is necessary to explicit which \textbf{linking rule} will be used to connect nodes with each other. As for the current simulation, the proper way to connect nodes cannot be based on a geometric rule (for instance, connecting nodes within a certain distance). Instead, it is appropriate to consider attractions as \textbf{access points} for the visitors' devices. Even with the \href{https://alchemistsimulator.github.io/reference/kdoc/alchemist/it.unibo.alchemist.model.implementations.linkingrules/-connect-to-access-point/index.html}{\texttt{ConnectToAccessPoint}} linking rule every node on the map will be connected with the rest of the network as the attractions are distributed throughout the map, and, working as access points, they can cover it without leaving connectionless areas.

\begin{lstlisting}[language=yaml, label=code:linking, caption=Defining the linking rule: only nodes with the molecule \texttt{attraction} will connect to other nodes within a radius of 100 meters.]
# The network model used allows to choose the nodes that have the
# "attraction" molecule in order to simulate an access point behaviour
network-model:
  type: ConnectToAccessPoint
  parameters: [100.0, "attraction"]
\end{lstlisting}

\subsection{Programmed Behaviours}
With the environment and the nodes set, the next step consists in programming their behaviours.

\subsection{Data Extraction}