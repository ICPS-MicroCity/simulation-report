\section{Conclusions}
This project was developed for the \textit{Intelligent Cyber-Physical Systems} course.
As already discussed in the introduction~\ref{sec:introduction}, sometimes it is not possible to implement complex systems due to uncertainty and limited resources.
In the case of cyber-physical systems and infrastructures, this aspect is even more evident.
Thus, the only way to demonstrate whether these systems can work or not is through proofs of concept.
The information extracted from the implemented simulations can be used to predict scenarios and outcomes 
under determined conditions and to compare different strategies for recommendations.
Even though the simulations do not perfectly match the reality, we still consider ourselves satisfied 
with the work done, aware of the fact that small improvements can help make the scenarios even more 
realistic.

One possible future development for the project is to switch language for the aggregate programming computations.
In fact, \textit{Alchemist} offers many \textit{incarnations} that allow developers to build complex systems with different syntaxs and paradigms.
For instance, a \textit{work-in-progress} version of the simulation is being developed with the~\href{https://scafi.github.io/}{\texttt{scafi}} incarnation.
In fact, \texttt{scafi} is a \textit{Domain Specific Language} for aggregate computing built on top of \textit{Scala}.
The advantages that it brings are a static type-checking system that makes it easier to develop complex simulations.
Moreover, it allows performing aggregate and local computations, that belong to different paradigms, with the same language and data structures.
